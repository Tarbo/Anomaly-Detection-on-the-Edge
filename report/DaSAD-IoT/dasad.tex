%% For double-blind review submission, w/o CCS and ACM Reference (max submission space)
\documentclass[10pt,sigplan,review,anonymous]{acmart}\settopmatter{printfolios=true,printccs=false,printacmref=false}
%% For double-blind review submission, w/ CCS and ACM Reference
%\documentclass[sigplan,review,anonymous]{acmart}\settopmatter{printfolios=true}
%% For single-blind review submission, w/o CCS and ACM Reference (max submission space)
%\documentclass[sigplan,review]{acmart}\settopmatter{printfolios=true,printccs=false,printacmref=false}
%% For single-blind review submission, w/ CCS and ACM Reference
%\documentclass[sigplan,review]{acmart}\settopmatter{printfolios=true}
%% For final camera-ready submission, w/ required CCS and ACM Reference
%\documentclass[sigplan]{acmart}\settopmatter{}


%% Conference information
%% Supplied to authors by publisher for camera-ready submission;
%% use defaults for review submission.
\acmConference[Middleware'19]{ACM/IFIP Middleware conference}{December 09--133, 
2019}{UC Davis, CA, USA}
\acmYear{2019}
\acmISBN{} % \acmISBN{978-x-xxxx-xxxx-x/YY/MM}
\acmDOI{} % \acmDOI{10.1145/nnnnnnn.nnnnnnn}
\startPage{1}

%% Copyright information
%% Supplied to authors (based on authors' rights management selection;
%% see authors.acm.org) by publisher for camera-ready submission;
%% use 'none' for review submission.
\setcopyright{none}
%\setcopyright{acmcopyright}
%\setcopyright{acmlicensed}
%\setcopyright{rightsretained}
%\copyrightyear{2018}           %% If different from \acmYear

%% Bibliography style
\bibliographystyle{ACM-Reference-Format}
%% Citation style
\citestyle{acmauthoryear}  %% For author/year citations
%\citestyle{acmnumeric}     %% For numeric citations
%\setcitestyle{nosort}      %% With 'acmnumeric', to disable automatic
                            %% sorting of references within a single citation;
                            %% e.g., \cite{Smith99,Carpenter05,Baker12}
                            %% rendered as [14,5,2] rather than [2,5,14].
%\setcitesyle{nocompress}   %% With 'acmnumeric', to disable automatic
                            %% compression of sequential references within a
                            %% single citation;
                            %% e.g., \cite{Baker12,Baker14,Baker16}
                            %% rendered as [2,3,4] rather than [2-4].


%%%%%%%%%%%%%%%%%%%%%%%%%%%%%%%%%%%%%%%%%%%%%%%%%%%%%%%%%%%%%%%%%%%%%%
%% Note: Authors migrating a paper from traditional SIGPLAN
%% proceedings format to PACMPL format must update the
%% '\documentclass' and topmatter commands above; see
%% 'acmart-pacmpl-template.tex'.
%%%%%%%%%%%%%%%%%%%%%%%%%%%%%%%%%%%%%%%%%%%%%%%%%%%%%%%%%%%%%%%%%%%%%%


%% Some recommended packages.
\usepackage{booktabs}   %% For formal tables:
                        %% http://ctan.org/pkg/booktabs
\usepackage{subcaption} %% For complex figures with subfigures/subcaptions
                        %% http://ctan.org/pkg/subcaption


\begin{document}

%% Title information
\title[DADIE]{Distributed Anomaly Detection in Internet of Everything}         
%% [Short Title] is optional;
                                        %% when present, will be used in
                                        %% header instead of Full Title.
\titlenote{with title note}             %% \titlenote is optional;
                                        %% can be repeated if necessary;
                                        %% contents suppressed with 'anonymous'
\subtitle{Subtitle}                     %% \subtitle is optional
\subtitlenote{with subtitle note}       %% \subtitlenote is optional;
                                        %% can be repeated if necessary;
                                        %% contents suppressed with 'anonymous'


%% Author information
%% Contents and number of authors suppressed with 'anonymous'.
%% Each author should be introduced by \author, followed by
%% \authornote (optional), \orcid (optional), \affiliation, and
%% \email.
%% An author may have multiple affiliations and/or emails; repeat the
%% appropriate command.
%% Many elements are not rendered, but should be provided for metadata
%% extraction tools.

%% Author with single affiliation.
\author{Okwudili M. Ezeme}
%\authornote{with author1 note}          %% \authornote is optional;
                                        %% can be repeated if necessary
\orcid{000000009570566}             %% \orcid is optional
\affiliation{
  %\position{Position1}
  \department{Electrical, Computer and Software Engineering}              %% 
  %%\department is recommended
  \institution{Ontario Tech Universiy}            %% \institution is required
  %\streetaddress{Street1 Address1}
  \city{Oshawa}
  \state{ON}
  \postcode{L1H 7K4}
  \country{Canada}                    %% \country is recommended
}
\email{mellitus.ezeme@uoit.ca}          %% \email is recommended

%% Author with two affiliations and emails.
\author{Qusay H. Mahmoud}
%\authornote{with author2 note}          %% \authornote is optional;
                                        %% can be repeated if necessary
\orcid{0000000304725757}             %% \orcid is optional
\affiliation{
  %\position{Professor}
    %\position{Position1}
   \department{Electrical, Computer and Software Engineering}              %% 
   %%\department is recommended
   \institution{Ontario Tech Universiy}            %% \institution is required
   %\streetaddress{Street1 Address1}
   \city{Oshawa}
   \state{ON}
   \postcode{L1H 7K4}
   \country{Canada}                    %% \country is recommended
}
\email{qusay.mahmoud@uoit.ca} 

\author{Akramul Azim}
%\authornote{with author2 note}          %% \authornote is optional;
%% can be repeated if necessary
\orcid{0000000262926939}             %% \orcid is optional
\affiliation{
	%\position{Professor}
	%\position{Position1}
	\department{Electrical, Computer and Software Engineering}              %% 
	%%\department is recommended
	\institution{Ontario Tech Universiy}            %% \institution is required
	%\streetaddress{Street1 Address1}
	\city{Oshawa}
	\state{ON}
	\postcode{L1H 7K4}
	\country{Canada}                    %% \country is recommended
}
\email{akramul.azim@uoit.ca} 
%% Abstract
%% Note: \begin{abstract}...\end{abstract} environment must come
%% before \maketitle command
\begin{abstract}
Our world has become so connected that the Internet of Things (IoT) has now 
become a subset of the Internet of Everything (IoE). While this increased 
connectivity has opened up a vista of opportunities for solutions to myriads of 
challenges plaguing our world, the increased connectivity has also increased 
our vulnerability to security breaches. Moreover, with the coming of 5G comes 
the dawn of connectivity explosion. 
Because security is not the primary function of most of these devices, they 
dedicate only spare system resources for monitoring and diagnosis. \par 
However, with the increased connectivity to each other and the cloud, a 
middleware can leverage a hybrid pool of resources to build a robust anomaly 
detection framework that can help connected powerful and less powerful devices 
to remain secure while performing their primary function. By using the kernel 
events generated from each process in each node, we provide a 
reverse-engineering monitoring process that also provides uniqueness in the 
distributed platform. A 
simulation environment with the middleware shows effective latency management 
for both time-constrained and non-time constrained flows without depleting the 
primary function of the devices.
\end{abstract}


%% 2012 ACM Computing Classification System (CSS) concepts
%% Generate at 'http://dl.acm.org/ccs/ccs.cfm'.
\begin{CCSXML}
<ccs2012>
<concept>
<concept_id>10011007.10011006.10011008</concept_id>
<concept_desc>Software and its engineering~General programming languages</concept_desc>
<concept_significance>500</concept_significance>
</concept>
<concept>
<concept_id>10003456.10003457.10003521.10003525</concept_id>
<concept_desc>Social and professional topics~History of programming languages</concept_desc>
<concept_significance>300</concept_significance>
</concept>
</ccs2012>
\end{CCSXML}

\ccsdesc[500]{Software and its engineering~General programming languages}
\ccsdesc[300]{Social and professional topics~History of programming languages}
%% End of generated code


%% Keywords
%% comma separated list
\keywords{keyword1, keyword2, keyword3}  %% \keywords are mandatory in final camera-ready submission


%% \maketitle
%% Note: \maketitle command must come after title commands, author
%% commands, abstract environment, Computing Classification System
%% environment and commands, and keywords command.
\maketitle


\section{Introduction}

Text of paper \ldots


%% Acknowledgments
\begin{acks}                            %% acks environment is optional
                                        %% contents suppressed with 'anonymous'
  %% Commands \grantsponsor{<sponsorID>}{<name>}{<url>} and
  %% \grantnum[<url>]{<sponsorID>}{<number>} should be used to
  %% acknowledge financial support and will be used by metadata
  %% extraction tools.
  This material is based upon work supported by the
  \grantsponsor{GS100000001}{National Science
    Foundation}{http://dx.doi.org/10.13039/100000001} under Grant
  No.~\grantnum{GS100000001}{nnnnnnn} and Grant
  No.~\grantnum{GS100000001}{mmmmmmm}.  Any opinions, findings, and
  conclusions or recommendations expressed in this material are those
  of the author and do not necessarily reflect the views of the
  National Science Foundation.
\end{acks}


%% Bibliography
%\bibliography{bibfile}


%% Appendix
\appendix
\section{Appendix}

Text of appendix \ldots

\end{document}
